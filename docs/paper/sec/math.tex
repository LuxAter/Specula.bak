\documentclass[../specula.tex]{subfiles}

\begin{document}
\section{Mathematics}%
\label{sec:mathematics}

Before any real work can be done, it is important to develop a strong linear
algebra and geometry library. This section skims over the details, as many
tools can be used to implement a linear algebra library, but there are some
necessities that our implementation depends upon, so we specify the
implementation details that are required by the reset of the system.

For \texttt{C++} a library that implements almost all of the requirements is
\texttt{glm}\footnote{https://github.com/g-truc/glm}. Although if one uses
this, it is important to be careful with the different types of vectors, and
template overloading can lead to unexpected issues.

\subsection{Real Numbers}%
\label{sub:Real Numbers}

The first component to implement is the real number type. The real number type
(or floating point type) defines the maxim accuracy of the renderer. For
example if the floating point type can only guarantee three digits of
precision, then any computations will have relatively large errors. We will
implement this type called \texttt{Float} like so.

\begin{minted}{c++}
  #ifdef SPECULA_DOUBLE_PRECISION
    typedef double Float
  #else
    typedef float Float
  #endif
\end{minted}

\end{document}
